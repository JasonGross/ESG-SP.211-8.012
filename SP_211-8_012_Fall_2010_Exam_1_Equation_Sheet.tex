\documentclass{article}
\usepackage[T1]{fontenc}
\usepackage[margin=1in]{geometry}
\usepackage{amsmath,amssymb}
\usepackage{mathtools}
\usepackage{sectsty}
\sectionfont{\large}
\let\oldvec=\vec
\let\oldhat=\hat
\renewcommand{\vec}[1]{\oldvec{\boldsymbol{#1}}}
\renewcommand{\hat}[1]{\oldhat{\boldsymbol{#1}}}
\newcommand{\uvec}[1]{\hat{#1}}
\pagestyle{empty}
\begin{document}
{\large
\begin{center}
\textbf{\textsc{Massachusetts Institute of Technology}} \\
 \textbf{Department of Physics} \\
 \textbf{Experimental Study Group}
\end{center}
\noindent \textbf{Physics 8.012 \hfill Fall Term 2010}} \par

\section*{Kinematics in Polar Coordinates}
\begin{align*}
  \vec r & = r \uvec r &
  \vec v & = \dot r\uvec r + r\dot\theta \uvec\theta &
  \vec a & = \left(\ddot r - r(\dot \theta)^2\right)\uvec r + \left(2\dot r\dot \theta + r \ddot \theta\right)\uvec \theta \\
  \\
  \vec r & = r \uvec r &
  \vec v & = \frac{dr}{dt}\uvec r + r\frac{d\theta}{dt} \uvec\theta &
  \vec a & = \left(\frac{d^2r}{dt^2} - r\left(\frac{d\theta}{dt}\right)^2\right)\uvec r + \left(2\frac{dr}{dt}\frac{d\theta}{dt} + r \frac{d^2\theta}{dt^2}\right)\uvec \theta
\end{align*}

%\section*{Newton's Second Law}
%\begin{align*}
  %\vec F & = m \vec a = \frac{d}{dt}(m \vec v) & \vec F^\text{ext} & = \frac{d\vec p_\text{sys}}{dt}
%\end{align*}
%
%\section*{Torque and Angular Momentum}
%\begin{align*}
  %\vec \tau_S & = \sum_{i=1}^N \vec r_{S, i} \times \vec F_i &
    %\vec L_S & = \sum_{i = 1}^N \vec r_{S, i} \times m_i \vec v_i &
      %\vec \tau_S & = \frac{d\vec L_S}{dt} \\
  %\vec \tau_S & =\vec r_{S, F} \times \vec F + \vec \tau_\text{cm} &
    %\vec L_S & = \vec r_{S, \text{cm}} \times m\vec v_\text{cm} + \vec L_\text{cm}
%\end{align*}
%
%\section*{Principle Axes}
%\begin{align*}
  %\vec \omega & = \vec \omega_1 + \vec \omega_2 + \vec \omega_3 \\
  %\vec L_\text{cm} & = I_1 \vec \omega_1 + I_2 \vec \omega_2 + I_3 \vec \omega_3 \\
  %K_\text{cm} & = \frac{1}{2} I_1 \omega_1^2 + \frac{1}{2} I_2 \omega_2^2 + \frac{1}{2} I_3 \omega_3^2
%\end{align*}
%
%\section*{Energy}
%\begin{align*}
  %K & = \frac{1}{2} m v_\text{cm}^2 + K_\text{cm} &
    %W_{AB} & \equiv \smashoperator{\int}_A^B \vec F \cdot d\vec s = \Delta K \\
  %U(B) - U(A) & = -\smashoperator{\int}_A^B \vec F_c \cdot d\vec s &
    %W_\text{nc} & = \Delta K + \Delta U \equiv \Delta E
%\end{align*}

\end{document}
