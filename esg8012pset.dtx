% \iffalse meta-comment
%
% Copyright (C) 2010 by Jason Gross (jgross@mit.edu)
%
% This file may be distributed and/or modified under the
% conditions of the LaTeX Project Public License, either
% version 1.3c of this license or (at your option) any later
% version. The latest version of this license is in:
%
%   http://www.latex-project.org/lppl.txt
%
% and version 1.3c or later is part of all distributions of
% LaTeX version 2005/12/01 or later.
%
% \fi
%
%
% \iffalse
%<class>\NeedsTeXFormat{LaTeX2e}[1999/12/01]
%<driver>\ProvidesFile{esg8012pset.dtx}
%<class>\ProvidesClass{esg8012pset}
%<class> [2010/10/08 v0.1b ESG PSet Template]
%<class>\LoadClass[notitlepage,11pt,twoside,letterpaper]{article}
%<class>\RequirePackage{fancyhdr}
%<class>\RequirePackage[T1]{fontenc}
%<class>\RequirePackage{lastpage}
%<class>\RequirePackage{xifthen}
%<class>\RequirePackage{verbatim}
%<class>\RequirePackage[margin=1in]{geometry}
%<class>\RequirePackage{etoolbox,etextools}
%
%<class>\newboolean{esg8012pset@solutions}\newboolean{esg8012pset@problems}
%<class>\newboolean{esg8012pset@pdfproblems}\newboolean{esg8012pset@pdfsolutions}
%<class>\DeclareOption{problems}{\setboolean{esg8012pset@problems}{true}\setboolean{esg8012pset@solutions}{false}}
%<class>\DeclareOption{solutions}{\setboolean{esg8012pset@problems}{false}\setboolean{esg8012pset@solutions}{true}}
%<class>\DeclareOption{makeproblemspdf}{\setboolean{esg8012pset@pdfproblems}{true}}
%<class>\DeclareOption{makesolutionspdf}{\setboolean{esg8012pset@pdfsolutions}{true}}
%<class>\DeclareOption{makeallpdfs}{\setboolean{esg8012pset@pdfproblems}{true}\setboolean{esg8012pset@pdfsolutions}{true}}
%<class>\ProcessOptions\relax
%
%<*driver>
\documentclass{ltxdoc}
\EnableCrossrefs
\CodelineIndex
\RecordChanges
\begin{document}
  \DocInput{esg8012pset.dtx}
\end{document}
%</driver>
% \fi
%
% \CheckSum{0}
%
% \CharacterTable
% {Upper-case    \A\B\C\D\E\F\G\H\I\J\K\L\M\N\O\P\Q\R\S\T\U\V\W\X\Y\Z
%  Lower-case    \a\b\c\d\e\f\g\h\i\j\k\l\m\n\o\p\q\r\s\t\u\v\w\x\y\z
%  Digits        \0\1\2\3\4\5\6\7\8\9
%  Exclamation   \!     Double quote  \"     Hash (number) \#
%  Dollar        \$     Percent       \%     Ampersand     \&
%  Acute accent  \'     Left paren    \(     Right paren   \)
%  Asterisk      \*     Plus          \+     Comma         \,
%  Minus         \-     Point         \.     Solidus       \/
%  Colon         \:     Semicolon     \;     Less than     \<
%  Equals        \=     Greater than  \>     Question mark \?
%  Commercial at \@     Left bracket  \[     Backslash     \\
%  Right bracket \]     Circumflex    \^     Underscore    \_
%  Grave accent  \`     Left brace    \{     Vertical bar  \|
%  Right brace   \}     Tilde         \~}
%
% \changes{v0.1}{2010/09/07}{Initial version.}
% \changes{v0.1b}{2010/10/08}{Made title textsc.}
%
% \GetFileInfo{esg8012pset.dtx}
%
% \DoNotIndex{\#,\$,\%,\&,\@,\\,\{,\},\^,\_,\~,\ }
% \DoNotIndex{}
%
% \title{The \textsf{esg8012pset} class\thanks{This document
%   corresponds to \textsf{esg8012pset}~\fileversion,
%   dated~\filedate.}}
% \author{Jason Gross \\ \texttt{jgross@mit.edu}}
%
% \maketitle
%
% \section{Introduction}
% 
% The \textsf{esg8012pset} class provides a template for ESG class PSets.
%
% \section{Usage}
% I give the usage and specification of every macro defined.  I give bugs when
% I know them (please email me if you find other bugs, or have fixes for the
% bugs I list).  I sometimes give extra description or justification.
%
% \DescribeMacro{\duedate}
% \noindent Usage: |\duedate|\marg{date} \par\noindent
% Specification: The \meta{date} is used as the due date.
%
% \StopEventually{\PrintChanges\PrintIndex}
%
% \section{Setup}
%    \begin{macrocode}
\ifthenelse{\boolean{esg8012pset@problems} \OR \boolean{esg8012pset@solutions}}{
}{
  \expandnext{\renewcommand{\end}[1]}{\end{#1}\ifcsname end#1@hook\endcsname\csname end#1@hook\endcsname\else\fi}

  \newcommand{\AfterEnvironment}[2]{%
    \ifcsname end#1@hook\endcsname
    \else
      \csdef{end#1@hook}{}%
    \fi
    \csappto{end#1@hook}{#2}%
  }


  \newwrite\esgpset@problemsout
  \newwrite\esgpset@solutionsout
  \newwrite\esgpset@tempout
  \newcommand{\esgpset@compilefile}[1]{\write18{pdflatex "#1"}}
  \edef\esgpset@problemsfilename{\jobname\string_Problems.tex}
  \edef\esgpset@solutionsfilename{\jobname\string_Solutions.tex}
  \edef\esgpset@tempfilename{\jobname.tmp}
  \newcommand{\writetoboth}[1]{\writetoproblems{#1}%
    \writetosolutions{#1}}
  \newcommand{\writetoall}[1]{\writetoboth{#1}\writetothis{#1}}
  \newcommand{\writetoproblems}[1]{\immediate\write\esgpset@problemsout{#1}}
  \newcommand{\writetosolutions}[1]{\immediate\write\esgpset@solutionsout{#1}}
  \newcommand{\writetothis}[1]{{\edef\temp{#1}\expandafter}\expandafter\scantokens\expandafter{\temp}}

  \immediate\openout\esgpset@problemsout\esgpset@problemsfilename
  \immediate\openout\esgpset@solutionsout\esgpset@solutionsfilename

  \AtEndDocument{
    \writetoboth{\string\end{document}}
    \immediate\closeout\esgpset@problemsout
    \immediate\closeout\esgpset@solutionsout
    \ifthenelse{\boolean{esg8012pset@pdfsolutions}}{\esgpset@compilefile{\esgpset@solutionsfilename}}{}
    \ifthenelse{\boolean{esg8012pset@pdfproblems}}{\esgpset@compilefile{\esgpset@problemsfilename}}{}
  }

  \begingroup
    \writetosolutions{%
      \string\documentclass[solutions]{esg8012pset}
    }
    \writetoproblems{%
      \string\documentclass[problems]{esg8012pset}
    }
  \endgroup

  \newenvironment{preamble}{%
    \begingroup% Lets Keep the Changes Local
      \immediate\openout\esgpset@tempout\esgpset@tempfilename
      \@bsphack
      \let\do\@makeother\dospecials\catcode`\^^M\active
      \def\verbatim@processline{\writetoboth{\the\verbatim@line}\immediate\write\esgpset@tempout{\the\verbatim@line}}
      \verbatim@start
  }{\@esphack\immediate\closeout\esgpset@tempout\endgroup}
  \AfterEnvironment{preamble}{\input{\esgpset@tempfilename}}

  \AtBeginDocument{

    \begingroup
      \writetoboth{%
        \string\classname{\expandafter\unexpanded\expandafter{\@classname}}
      }
      \writetoboth{%
        \string\semester{\expandafter\unexpanded\expandafter{\@semester}}
      }
      \writetoboth{%
        \string\problemsetnumber{\expandafter\unexpanded\expandafter{\@problemsetnumber}}
      }
      \writetoboth{%
        \string\date{\expandafter\unexpanded\expandafter{\@date}}
      }
      \writetoboth{%
        \string\duedate{\expandafter\unexpanded\expandafter{\@duedate}}
      }
      \writetoboth{%
        \string\readingassignment{\expandafter\unexpanded\expandafter{\@readingassignment}}
      }
      \writetoboth{\string\begin{document}}
    \endgroup
  }
}


\pagestyle{fancy}
\headheight 14.5pt
\fancyhead{}
\fancyfoot{}
\cfoot{\thepage\space of \pageref{LastPage}} 

\let\@seccntformat\@gobble

\AtBeginDocument{
  \begingroup
    \ifthenelse{\boolean{esg8012pset@problems}}{%
      \edef\@cheader{Problem Set \@problemsetnumber\space - Problems}
    }{
      \ifthenelse{\boolean{esg8012pset@solutions}}{
        \edef\@cheader{Problem Set \@problemsetnumber\space - Solutions}
      }{
        \edef\@cheader{Problem Set \@problemsetnumber\space - Problems}
      }
    }
  \expandafter\endgroup
  \expandafter\chead\expandafter{\@cheader}
  \begingroup
    \bf
    \begin{center}%
      {\noindent  \textsc{Massachusetts Institute of Technology} \par}%
      {\noindent  Experimental Study Group \par}%
    \end{center}%
    {\noindent  \@classname, \@semester \par}%
    \begin{center}%
      {\noindent \Large
        Problem Set \@problemsetnumber
        \ifthenelse{\boolean{esg8012pset@solutions}}{% \OR \NOT \boolean{esg8012pset@problems}{%
          \space Solutions%
        }{}%
      \par}%
    \end{center}%
    {\noindent Due: \@duedate \\\\}%
    {\noindent Reading: \@readingassignment \par}%
  \endgroup
  \global\let\duedate\relax
  \global\let\problemsetnumber\relax
  \global\let\semester\relax
  \global\let\classname\relax
  \global\let\readingassignment\relax
  \global\let\@duedate\relax
  \global\let\@problemsetnumber\relax
  \global\let\@semester\relax
  \global\let\@classname\relax
  \global\let\@readingassignment\relax
}
%    \end{macrocode}
%
% \begin{macro}{\duedate}
% \begin{macro}{\problemsetnumber}
% \begin{macro}{\semester}
% \begin{macro}{\classname}
% \begin{macro}{\readingassignment}
%    These four macros are provided by \file{esg8012pset.dtx} to provide
%    information about the class assigning the pset.
%    The information is stored away in internal control sequences.
%    It is the task of the |\maketitle| command to use the
%    information provided. The definitions of these macros are shown
%    here for information.
%    \begin{macrocode}
\newcommand*{\duedate}[1]{\gdef\@duedate{#1}}
\newcommand*{\problemsetnumber}[1]{\gdef\@problemsetnumber{#1}}
\newcommand*{\semester}[1]{\gdef\@semester{#1}}
\newcommand*{\classname}[1]{\gdef\@classname{#1}}
\newcommand*{\readingassignment}[1]{\gdef\@readingassignment{#1}}
%    \end{macrocode}
% \end{macro}
% \end{macro}
% \end{macro}
% \end{macro}
% \end{macro}
%
%
%
%
% \subsection{Problem Environments}
% \begin{environment}{problem}
% \begin{environment}{statement}
% \begin{environment}{solution}
%    \begin{macrocode}
\newenvironment{problem}[2][\relax]{%
  \ifthenelse{\equal{#1}{\relax}}{%
    \writetoall{\string\section{Problem \string\thesection: #2}}%
  }{%
    \writetoall{\string\section*{Problem #1: #2}}%
  }%
  \writetosolutions{\string\subsection{Problem}}%
  \begingroup% Lets Keep the Changes Local
    \@bsphack
    \immediate\openout\esgpset@tempout\esgpset@tempfilename
    \let\do\@makeother\dospecials\catcode`\^^M\active
    \def\verbatim@processline{\writetoboth{\the\verbatim@line}\immediate\write\esgpset@tempout{\the\verbatim@line}}%
    \verbatim@start
}{\@esphack\immediate\closeout\esgpset@tempout\endgroup
  \input{\esgpset@tempfilename}%
}
\newenvironment{solution}{%
  \writetosolutions{\string\subsection{Solution}}%
  \begingroup% Lets Keep the Changes Local
    \@bsphack
    \let\do\@makeother\dospecials\catcode`\^^M\active
    \def\verbatim@processline{\writetosolutions{\the\verbatim@line}}%
    \verbatim@start
}{\@esphack\endgroup}% 
%    \end{macrocode}
% \end{environment}
% \end{environment}
% \end{environment}
%
%
%
% \Finale
\endinput
